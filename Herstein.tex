\documentclass[10pt,a4paper,oneside]{article}
\usepackage[utf8x]{inputenc}
\usepackage[activeacute,spanish]{babel}
\usepackage{amsthm}
\usepackage{amssymb}
\usepackage{graphicx}
\usepackage{wrapfig}
\usepackage{color}
\usepackage{textpos}
\usepackage[T1]{fontenc}

\title{Herstein Algebra Moderna\\Resuelto} % Título
\author{Humberto Alonso Villegas} %Autor.+ agregando \and Otro autor
\date{\today} % Fecha siempre actualizada al día presente al compilar.

\begin{document} % Inicio del documento
	\maketitle % Hace la portada
	\newpage
	
	\chapter{Nociones Preliminares}
		\section[''Teoría de Conjuntos'']{Teoría de Conjuntos}
			\textbf{Ejemplo 1.-} Sea s un conjunto cualquiera y definamos ~ en S por a ~ b para $a, b \in S $ si y solo si a = b. Hemos definido claramente, así, una relación de equivalencia sobre S. En 
			
			\begin{itemize}
			\item pensar como humano
			\item actuar como humano
			\item actuar racionalmente
			\item pensar racionalmente
			\end {itemize}
			\newpage
	\newpage



	\chapter{Teoría de Grupos}
		\section["Teoría de"]{Teoría de Grupos}
			\subsection{Definición de Grupo}
				\textbf{Ejemplo 1.-} Supongamos que $G = \mathbb{Z}$, con a·b, para $a,b\in G$, definida como la suma usual entre enteros, es decir, con $a·b=a+b$. Demostrar que G es un grupo abeliano infinito en el que 0 juega el papel de e y $-a$ el de $a^{-1}$
				G es un grupo $\iff$ cumple lo siguiente.
				\begin{enumerate}
					\item $\forall$ a,b $\in$ G $\in$ G a·b $\in$ G
					\item $\forall$ a,b,c $\in$ G, a·(b·c) = (a·b)·c
					\item $\exists$ e $\in$ G : $\forall$ a $\in$ G, e·a=a·e=a
					\item $\forall$ a $\in$ G $\exists$ $a^{-1} \in$ G : a·$a^{-1}$ = e
				\end{enumerate}

				%\begin{proof}[Demostración.]
				%	\
				%\end{proof} 
				\begin{proof}:
					\newline
					\newline
					\textbf{1.-} Sean a,b $\in$ G, a·b $\in$ G $\iff$ a+b $\in \mathbb{Z} \iff$ a,b $\in \mathbb{Z}$
					\newline
					\newline
					\textbf{2.-} Sean a,b,c $\in$ G, $\Rightarrow$ a,b,c $\in$ G, $\Rightarrow$ a·(b·c) = a+(b+c) = (a+b)+c = (a·b)·c  $\Rightarrow$ a·(b·c)=(a·b)·c
					\newline
					\newline
					\textbf{3.-} Sea a $\in$ G, $\exists$ e $\in$ G : e·a = a·e = a $\forall$ a $\in$ G $\iff$  $\exists$ w $\in$ $\mathbb{Z}$ : w·a = a·w = a $\forall$ a $\in$ $\mathbb{Z}$ (1 cumple)
					\newline
					\newline
				\end{proof}

\end{document} % Fin del documento.