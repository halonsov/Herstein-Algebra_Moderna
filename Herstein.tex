\documentclass[10pt,a4paper,oneside]{article}
\usepackage[utf8x]{inputenc}
\usepackage[activeacute,spanish]{babel}
\usepackage{amsthm}
\usepackage{amssymb}
\usepackage{graphicx}
\usepackage{wrapfig}
\usepackage{color}
\usepackage{textpos}
\usepackage[T1]{fontenc}

\title{Herstein Algebra Moderna\\Resuelto} % Título
\author{Humberto Alonso Villegas} %Autor.+ agregando \and Otro autor
\date{\today} % Fecha siempre actualizada al día presente al compilar.

\begin{document} % Inicio del documento
	\maketitle % Hace la portada
	\newpage
	
	\chapter{Nociones Preliminares}
		\section[''Teoría de Conjuntos'']{Teoría de Conjuntos}
			\textbf{Ejemplo 1.-} Sea s un conjunto cualquiera y definamos ~ en S por a ~ b para $a, b \in S $ si y solo si a = b. Hemos definido claramente, así, una relación de equivalencia sobre S. En 
			
			\begin{itemize}
			\item pensar como humano
			\item actuar como humano
			\item actuar racionalmente
			\item pensar racionalmente
			\end {itemize}
			\newpage
	\newpage



	\chapter{Teoría de Grupos}
		\section["Teoría de"]{Teoría de Grupos}
			\subsection{Definición de Grupo}
				\textbf{Ejemplo 1.-} Supongamos que $G = \mathbb{Z}$, con a·b, para $a,b\in G$, definida como la suma usual entre enteros, es decir, con $a·b=a+b$. Demostrar que G es un grupo abeliano infinito en el que 0 juega el papel de e y $-a$ el de $a^{-1}$
				G es un grupo $\iff$ cumple lo siguiente.
				\begin{enumerate}
					\item $\forall$ a,b $\in$ G $\in$ G a·b $\in$ G
					\item $\forall$ a,b,c $\in$ G, a·(b·c) = (a·b)·c
					\item $\exists$ e $\in$ G : $\forall$ a $\in$ G, e·a=a·e=a
					\item $\forall$ a $\in$ G $\exists$ $a^{-1} \in$ G : a·$a^{-1}$ = e
				\end{enumerate}

				%\begin{proof}[Demostración.]
				%	\
				%\end{proof} 
				\begin{proof}:
					\newline
					\newline
					\textbf{1.-} Sean a,b $\in$ G, a·b $\in$ G $\iff$ a+b $\in \mathbb{Z} \iff$ a,b $\in \mathbb{Z}$
					\newline
					\newline
					\textbf{2.-} Sean a,b,c $\in$ G, $\Rightarrow$ a,b,c $\in$ G, $\Rightarrow$ a·(b·c) = a+(b+c) = (a+b)+c = (a·b)·c  $\Rightarrow$ a·(b·c)=(a·b)·c
					\newline
					\newline
					\textbf{3.-} Sea a $\in$ G, $\exists$ e $\in$ G : e·a = a·e = a $\forall$ a $\in$ G $\iff$  $\exists$ w $\in$ $\mathbb{Z}$ : w·a = a·w = a $\forall$ a $\in$ $\mathbb{Z}$ (1 cumple)
					\newline
					\newline
					\textbf{4.-} Sea a $\in$ $\mathbb{Z}$, $\exists$ $a^{-1}$ $\in$ G : a·$a^{-1}$ = $a^{-1}$·a = e $\iff$ $\exists$ $a^{-1}$ $\in$ $\mathbb{Z}$ : a+$a^{-1}$ = $a^{-1}$+a = 1 (cumple -a)
					\newline
					\newline
					De esto se tiene que G es un grupo, ahora veamos que G es grupo abeliano
					\newline
					\newline
					Sea a,b $\in$ G $\Rightarrow$ a,b $\in$ $\mathbb{Z}$, a·b = b·a $\iff$ a+b = b+a 
				\end{proof}.
				\newline
				\newline
				\textbf{2.-} Supongamos que G consiste en los números reales 1 y -1 con la multiplicación entre números reales como operación. G es entonces un grupo abeliano de orden 2.
				\begin{proof}:
					\newline
					\newline
					Es claro que el orden de G es 2
					\newline
					\textbf{1, 3, 4}:
					\newline
					 1·1 = 1 $\in$ G, 1·(-1) = (-1)·1 =  -1 $\in$ G, (-1)·(-1) = 1 $\therefore$ tenemos que $\forall$ a,b $\in$ G, a·b $\in$ G, $\forall$ a $\in$ G $\exists$ $a^{-1}$ $\in$ G : a·$\a^{a-1}$ = e, $\exists$ e $in$ G : $\forall$ a $\in$ G a·e = a. Además lo anterior muestra que G es conmutativo
					 \newline
					 \newline
					 \textbf{2.-} Sean a,b,c en G, $\Rightarrow$ a,b,c $\in$ $\mathbb{R}$ $\therefore$ a·(b·c) = (a·b)·c
				\end{proof}.
			\newline
			\newline
			\textbf{3.-} Sea G = $S_{3}$, el grupo de todas las aplicaciones biyectivas del conjunto A = {$x_{1}$, $x_{2}$, $x_{3}$} sobre si mismo, con el producto, la composición. G es un grupo de orden 6.
			\begin{proof}:
				\newline
				\newline
				$\varphi_{e}$ := G $\rightarrow$ G donde:
				\newline
				$\varphi_{e}$($x_{1}$) = $x_{1}$
				\newline
				$\varphi_{e}$($x_{2}$) = $x_{2}$
				\newline
				$\varphi_{e}$($x_{3}$) = $x_{3}$
				\newline
				\newline
				$\varphi_{1}$ := G $\rightarrow$ G donde:
				\newline
				$\varphi_{1}$($x_{1}$) = $x_{1}$
				\newline
				$\varphi_{1}$($x_{2}$) = $x_{3}$
				\newline
				$\varphi_{1}$($x_{3}$) = $x_{2}$
				\newline
				\newline
				$\varphi_{2}$ := G $\rightarrow$ G donde:
				\newline
				$\varphi_{2}$($x_{1}$) = $x_{3}$
				\newline
				$\varphi_{2}$($x_{2}$) = $x_{2}$
				\newline
				$\varphi_{2}$($x_{3}$) = $x_{1}$
				\newline
				\newline
				$\varphi_{3}$ := G $\rightarrow$ G donde:
				\newline
				$\varphi_{3}$($x_{1}$) = $x_{2}$
				\newline
				$\varphi_{3}$($x_{2}$) = $x_{1}$
				\newline
				$\varphi_{3}$($x_{3}$) = $x_{3}$
				\newline
				\newline
				$\varphi_{4}$ := G $\rightarrow$ G donde:
				\newline
				$\varphi_{4}$($x_{1}$) = $x_{2}$
				\newline
				$\varphi_{4}$($x_{2}$) = $x_{3}$
				\newline
				$\varphi_{4}$($x_{3}$) = $x_{1}$
				\newline
				\newline
				$\varphi_{5}$ := G $\rightarrow$ G donde:
				\newline
				$\varphi_{5}$($x_{1}$) = $x_{3}$
				\newline
				$\varphi_{5}$($x_{2}$) = $x_{1}$
				\newline
				$\varphi_{5}$($x_{3}$) = $x_{2}$
				\newline
				\newline
				\textbf{1.-}:
				Sean $\varphi_{a}$ y  $\varphi_{b}$ $\in$ G y $\varphi_{C}$ = $\varphi_{a}\circ\varphi_{b}$, sabemos que $\varphi_{a}$ y $\varphi_{b}$ son aplicaciones biyectivas de A en A, $\therefore$ $\varphi_{C}$ también es una aplicación biyectiva de A en A $\therefore$ $\varphi_{C}$ $\in$ G
				\newline
				\newline
				\textbf{2.-}
				Veamos que $\varphi{a}\circ$($\varphi_{b}\circ\varphi_{c}$) = ($\varphi{a}\circ\varphi_{b}$)$\circ\varphi_{c}$ $\forall$ $\varphi{a}$,$\varphi_{b}$,$\varphi_{c}$ $\in$ G
				\newline
				Omitiremos cuando alguna $\varphi$ es $\varphi_{e}$, pues es claro que se cumple.
				\newline
				$\varphi{1}\circ$($\varphi_{1}\circ\varphi_{1}$) = $\varphi_{1}\circ\varphi_{e}$ = $\varphi_{1}$
				\newline
				($\varphi{1}\circ\varphi_{1}$)$\circ\varphi_{1}$ = $\varphi_{e}\circ\varphi_{1}$ = $\varphi_{1}$
				\newline
				$\varphi{2}\circ$($\varphi_{2}\circ\varphi_{2}$) = $\varphi_{2}\circ\varphi_{e}$ = $\varphi_{2}$
				\newline
				($\varphi{2}\circ\varphi_{2}$)$\circ\varphi_{2}$ = $\varphi_{e}\circ\varphi_{2}$ = $\varphi_{2}$
				\newline
				$\varphi{3}\circ$($\varphi_{3}\circ\varphi_{3}$) = $\varphi_{3}\circ\varphi_{e}$ = $\varphi_{3}$
				\newline
				($\varphi{3}\circ\varphi_{3}$)$\circ\varphi_{3}$ = $\varphi_{e}\circ\varphi_{3}$ = $\varphi_{3}$
				\newline
				$\varphi{4}\circ$($\varphi_{4}\circ\varphi_{4}$) = $\varphi_{4}\circ\varphi_{5}$ = $\varphi_{e}$
				\newline
				($\varphi{4}\circ\varphi_{4}$)$\circ\varphi_{4}$ = $\varphi_{5}\circ\varphi_{4}$ = $\varphi_{e}$
				\newline
				$\varphi{5}\circ$($\varphi_{5}\circ\varphi_{5}$) = $\varphi_{5}\circ\varphi_{e}$ = $\varphi_{5}$
				\newline
				($\varphi{5}\circ\varphi_{5}$)$\circ\varphi_{5}$ = $\varphi_{5}\circ\varphi_{e}$ = $\varphi_{5}$
				\newline
				\textbf{3.-}
				Es claro que $\varphi_{e}$ cumple $\forall$ $\varphi_{a}$ $\in$ G, $\varphi_{e}\circ\varphi_{a}$ = $\varphi_{a}\circ\varphi_{e}$ = a (con la composición como producto)
				\newline
				\newline
				\textbf{4.-}
				Sea $\varphi_{a}$ $\in$ G, $\therefore$ $\varphi_{a}$ es una aplicación biyectiva de A en A, $\therefore$ $\exists$ $\varphi_{a}^{-1}$ : $\varphi_{a}\circ\varphi_{a}^{-1}$ = $\varphi_{I}$. $\varphi_{a}^{-1}$ también es una aplicación biyectova de A en A, $\therefore$ $\varphi_{a}^{-1}$ $\in$ G
				\newline

			\end{proof}

\end{document} % Fin del documento.

1.-
	a->a
	b->b
	c->c
2.-
	a->a
	b->c
	c->b
3.-
	a->b
	b->a
	c->c
4.-
	a->c
	b->b
	c->a
5.-
	a->b
	b->c
	c->a
6.-
	a->c
	b->a
	c->b